%%%%%%%%%%%%%%%%%%%%%%%%%%%%%%%%%%%%%%%%%%%%%%%%%%%%%%%%%%%%%%%%%%%%%%%%
% This is the conclusion chapter file.
%%%%%%%%%%%%%%%%%%%%%%%%%%%%%%%%%%%%%%%%%%%%%%%%%%%%%%%%%%%%%%%%%%%%%%%%
%
% Author:   René Widmer
%           Institute for Surgical Technology and Biomechanics ISTB
%           University of Bern
%           rene.widmer@istb.unibe.ch
%
% Date:     10/28/2009
%
%%%%%%%%%%%%%%%%%%%%%%%%%%%%%%%%%%%%%%%%%%%%%%%%%%%%%%%%%%%%%%%%%%%%%%%%

\begin{abstract}

    Accurate detection of extracapsular extension (ECE) is crucial for improving prognosis and guiding treatment decisions in head and neck cancers. Traditional methods relying on physician expertise to interpret contrast-enhanced CT images often lead to significant variability and inconsistencies. This study addresses these challenges by employing a multimodal approach that combines Fluorodeoxyglucose positron emission tomography (FDG-PET) and computed tomography (CT) imaging, utilizing advancements in deep learning to enhance ECE detection.

Building on the HECKTOR Challenge framework, we implemented a 3D U-Net architecture to evaluate segmentation performance under real-world conditions. Our primary objectives were to assess segmentation accuracy using the Dice similarity coefficient and to compare the model's performance with expert physician assessments. Utilizing the HECKTOR Challenge 2022 dataset, which encompasses 524 imaging cases from seven clinical centers, we preprocessed the images by resampling PET data and aligning modalities to a uniform isotropic voxel size. Data normalization and augmentation were conducted using the MONAI framework. The Dynamic UNet (DynUNet) architecture generated binary outputs for background, primary tumors, and nodal tumors, employing the AdamW optimizer and a hybrid loss function combining Dice and cross-entropy losses. A five-fold cross-validation strategy enhanced model generalization, while a test set of 50 cases underwent evaluations with various perturbations to assess model robustness.

The analysis evaluated the effects of different perturbations on segmentation performance, focusing on metrics such as the Dice coefficient, Hausdorff distance, sensitivity, specificity, and accuracy. Significant performance declines were noted for both nodal and primary tumor labels under spike, bias, and noise perturbations, while the model demonstrated stability against motion, blur, and ghost perturbations. Strong correlations were observed between the Dice coefficient and clinical evaluations, particularly for nodal labels, where the first observer achieved a peak correlation of 0.901. This suggests that improved Dice coefficients correspond to more accurate clinical ratings, reinforcing the model's relevance in detecting ECE.

Clinical evaluations showed robust inter-observer agreement for nodal labels, with the first observer rating approximately 75\% of original cases between 3 and 4 stars, indicating a tendency toward positive assessments. The second observer, however, assigned lower ratings, reflecting a more critical evaluation approach, yet still showed a balanced distribution of scores. The significant correlation between the Dice coefficient and clinical ratings suggests that improving segmentation accuracy could lead to more consistent clinical assessments, ultimately supporting better patient management.

The perturbation analysis revealed that the segmentation model exhibited robustness to ghosting, blurring, and motion artifacts, with minimal variation in the Delta Dice coefficient (approximately 0.1) under severe conditions. However, the model showed decreased resilience to noise, particularly spike noise and bias, highlighting areas for improvement. The PET modality demonstrated inherent resilience due to its focus on functional rather than detailed anatomical information.

Limitations include the absence of MRI data, critical for diagnosing primary tumors, and reliance on the Hecktor dataset, which lacks consistent patient-specific information. Clinician disagreement with the ground truth may introduce bias in the Dice coefficient.

Despite these limitations, the model demonstrated promising performance with high metric scores across cases, reinforcing its clinical relevance. Future work should integrate additional imaging modalities, such as MRI and contrast-enhanced CT, refine data augmentation strategies to enhance robustness, and develop a classification model to evaluate the likelihood of ECE for improved diagnostic precision and patient management.\end{abstract}

\endinput
