%%%%%%%%%%%%%%%%%%%%%%%%%%%%%%%%%%%%%%%%%%%%%%%%%%%%%%%%%%%%%%%%%%%%%%%%
% This is the conclusion chapter file.
%%%%%%%%%%%%%%%%%%%%%%%%%%%%%%%%%%%%%%%%%%%%%%%%%%%%%%%%%%%%%%%%%%%%%%%%
%
% Author:   René Widmer
%           Institute for Surgical Technology and Biomechanics ISTB
%           University of Bern
%           rene.widmer@istb.unibe.ch
%
% Date:     10/28/2009
%
%%%%%%%%%%%%%%%%%%%%%%%%%%%%%%%%%%%%%%%%%%%%%%%%%%%%%%%%%%%%%%%%%%%%%%%%

\begin{abstract}

\noindent\textit{The abstract should provide a concise (300-400 word) summary of the motivation, methodology, main results and conclusions. For example:}

\vskip1em

Osteoporosis is a disease in which the density and quality of bone are reduced. As the bones become more porous and fragile, the risk of fracture is greatly increased. The loss of bone occurs progressively, often there are no symptoms until the first fracture occurs. Nowadays as many women are dying from osteoporosis as from breast cancer. Moreover it has been estimated that yearly costs arising from osteoporotic fractures alone in Europe worth 30 billion Euros.

Percutaneous vertebroplasty is the injection of bone cement into the vertebral body in order to relieve pain and stabilize fractured and/or osteoporotic vertebrae with immediate improvement of the symptoms. Treatment risks and complications include those related to needle placement, infection, bleeding and cement extravazation. The cement can leak into extraosseous tissues, including the epidural or paravertebral venous system eventually ending in pulmonary embolism and death.

The aim of this project was to develop a computational model to simulate the flow of two immiscible fluids through porous trabecular bone in order to predict the three-dimensional spreading patterns developing from the cement injection and minimize the risk of cement extravazation while maximizing the mechanical effect. The computational model estimates region specific porosity and anisotropic permeability from Hounsfield unit values obtained from patient-specific clinical computer tomography data sets. The creeping flow through the porous matrix is governed by a modified version of Darcy's Law, an empirical relation of the pressure gradient to the flow velocity with consideration of the complex rheological properties of bone cement.

To simulate the immiscible two phase fluid flow, i.e. the displacement of a biofluid by a biomaterial, a fluid interface tracking algorithm with mixed boundary representation has been developed. The nonlinear partial differential equation arising from the problem was numerically implemented into the open-source Finite Element framework \textit{libMesh}. The algorithm design allows the incorporation of the developed methods into a larger simulation of vertebral bone augmentation for pre-surgical planning.

First simulation trials showed close agreement with the findings from relevant literature. The computational model demonstrated efficiency and numerical stability. The future model development may incorporate the morphology of the region specific trabecular bone structure improving the models' accuracy or the prediction of the orientation and alignment of fiber-reinforced bone cements in order to increase fracture-resistance. 

\end{abstract}

\endinput
