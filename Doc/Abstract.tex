%%%%%%%%%%%%%%%%%%%%%%%%%%%%%%%%%%%%%%%%%%%%%%%%%%%%%%%%%%%%%%%%%%%%%%%%
% This is the conclusion chapter file.
%%%%%%%%%%%%%%%%%%%%%%%%%%%%%%%%%%%%%%%%%%%%%%%%%%%%%%%%%%%%%%%%%%%%%%%%
%
% Author:   René Widmer
%           Institute for Surgical Technology and Biomechanics ISTB
%           University of Bern
%           rene.widmer@istb.unibe.ch
%
% Date:     10/28/2009
%
%%%%%%%%%%%%%%%%%%%%%%%%%%%%%%%%%%%%%%%%%%%%%%%%%%%%%%%%%%%%%%%%%%%%%%%%

\begin{abstract}

Accurate detection of extracapsular extension (ECE) is crucial for improving prognosis and guiding treatment decisions in head and neck cancers. Traditional methods relying on physician expertise to interpret contrast-enhanced CT images often lead to significant variability. This study employs a multimodal approach combining FDG-PET and CT imaging, leveraging deep learning to enhance ECE detection, focusing on the segmentation phase.
\vskip1em
We implemented a 3D U-Net architecture to assess segmentation performance. Our objectives were to evaluate accuracy with the Dice similarity coefficient and compare the model's performance against expert assessments. We utilized the HECKTOR Challenge 2022 dataset, with additional preprocessing and data augmentation. The model generated binary outputs for background, primary tumors, and nodal tumors, employing a hybrid loss function that combined Dice and cross-entropy losses. A five-fold cross-validation strategy enhanced model generalization, and a test set of 50 cases was evaluated under various perturbations to assess robustness of segmentation performance across imaging modalities and tumor labels, focusing on key metrics.
\vskip1em
The results showed that the model was robust to ghosting, blurring, and motion artifacts, with minimal variation in these metrics even under severe conditions. However, the model demonstrated reduced resilience to noise, particularly spike noise and bias, revealing areas for improvement. These trends were consistent across both modalities and tumor labels.
\vskip1em
Clinical evaluations indicated strong inter-observer agreement, with a correlation of approximately 0.65 for both labels in cases without perturbations. Most cases received positive ratings, reflecting a general trend of agreement among observers. Significant correlations between the Dice coefficient and clinical evaluations, especially for primary tumors, suggest that higher Dice scores align with more accurate assessments. Notably, these correlations remained consistent and even improved under various perturbations, underscoring the relevance of the metric even in challenging conditions.
\vskip1em
Limitations include the absence of MRI data, important for diagnosing primary tumors, and reliance on the Hecktor dataset, which lacks consistent patient-specific details. Clinician disagreement with the ground truth could also introduce bias in the Dice coefficient. Despite these issues, the model performed well with strong metric scores, and with approximately 73\% and 80\% of nodal and primary cases, respectively, rated 3 or above on a 5-point scale supporting its clinical relevance. Future work should incorporate additional imaging modalities like MRI, refine data augmentation for improved robustness, and develop classification models to better predict ECE for enhanced diagnostic accuracy and patient care.
\end{abstract}

\endinput
