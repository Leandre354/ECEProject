%%%%%%%%%%%%%%%%%%%%%%%%%%%%%%%%%%%%%%%%%%%%%%%%%%%%%%%%%%%%%%%%%%%%%%%%
% This is the conclusion chapter file.
%%%%%%%%%%%%%%%%%%%%%%%%%%%%%%%%%%%%%%%%%%%%%%%%%%%%%%%%%%%%%%%%%%%%%%%%
%
% Author:   René Widmer
%           Institute for Surgical Technology and Biomechanics ISTB
%           University of Bern
%           rene.widmer@istb.unibe.ch
%
% Date:     10/28/2009
%
%%%%%%%%%%%%%%%%%%%%%%%%%%%%%%%%%%%%%%%%%%%%%%%%%%%%%%%%%%%%%%%%%%%%%%%%

\chapter{Discussion and Conclusions}

\section{Discussion}

\subsection{Robustness}

The results of the perturbation analysis demonstrate that the model exhibits robustness to approximately half of the applied perturbations, including ghosting, blurring, and motion artifacts with The Delta Dice coefficient exhibited a maximum variation of approximately 0.1 for the most severe cases assessed. During model training, various intensity augmentations were employed, such as Gaussian noise, smoothing, intensity shifts, and contrast adjustments. These augmentations were expected to enhance the model’s performance under these perturbations. However, it is important to note that these augmentations were applied exclusively to the CT modality, while the PET modality also demonstrated resilience to these perturbations.
This observed robustness of the PET modality to perturbations such as blur, motion, and ghosting could be attributed to its lower structural complexity compared to CT. PET imaging primarily provides functional information in the form of Standardized Uptake Value (SUV) heatmaps, which are inherently less reliant on fine anatomical details. As a result, PET images may exhibit a natural resistance to perturbations that distort or duplicate structural features, such as blur and ghosting, which typically affect high-resolution modalities like CT. Moreover, the lack of intricate anatomical structures in PET likely contributes to its relative insensitivity to motion artifacts, as these perturbations do not significantly compromise the functional information provided by the heatmaps.

Interestingly, despite the inclusion of noise augmentation during training, the model showed lower robustness to noise compared to the aforementioned perturbations. The model remained relatively robust at lower levels of noise severity, but performance degraded as the noise severity increased. On the other hand, perturbations such as spike noise and bias were particularly problematic, causing significant degradation in performance even at the lowest levels of severity with the Delta Dice coefficient approached values as high as 1 or near it.
\newpage
Consultation with a clinician revealed that the severity of the applied perturbations was higher than what is typically encountered in clinical practice. According to the clinician, such extreme perturbations would likely warrant image re-acquisition. This insight suggests that the model's demonstrated robustness to motion, blur, and ghosting artifacts is likely to hold in real-world scenarios, where such perturbations occur at lower levels of severity. Consequently, the model's performance under these conditions may be considered sufficient for practical clinical applications. Nevertheless, to improve the model’s robustness to these more severe perturbations, future training could incorporate augmentations involving bias, spike , and higher levels of noise severity. This approach could potentially enhance the model’s resilience to these challenging conditions.
\vskip1em
The analysis of spatial properties such as volume, surface area, boundary length, and entropy reveals an average correlation of approximately 0.3 with perturbations including spike, bias, and noise in nodal tumors. This observation suggests that larger and more complex regions are more susceptible to these perturbations. Conversely, a negative correlation was identified in PET imaging for both channels concerning motion and blur, with correlation values reaching as low as -0.56 for blur. Furthermore, the bias exhibited a negative correlation for the primary label in PET imaging, indicating that smaller and less complex primary tumors are disproportionately affected by these perturbations compared to nodal tumors.

Additionally, when perturbations are introduced in CT imaging, particularly for primary tumors, the variability and maximum standardized uptake value (SUVmax) in PET demonstrate stronger correlations than observed in scenarios where perturbations are applied directly to PET. 
This finding suggests an inherent relationship between the metabolic activity of tumors (from PET SUVmax) and their sensitivity to CT-based segmentation perturbations, highlighting how tumors with greater metabolic intensity tend to experience more segmentation deviations when the CT is altered.
In contrast, the intensity properties of CT do not yield significantly higher correlations with PET when subjected to perturbations compared to the effects observed when perturbations are directly applied to PET.

Moreover, focusing on PET variability, we observed a prominent negative correlation of approximately -0.45 for primary tumors under PET perturbations, indicating that the impact of blur is more pronounced in less variable signals. 
This makes sense because uniform metabolic activity often correlates with a simpler, smoother anatomical structure. As a result, these tumors are less sensitive to the smoothing effect of blurring, since their boundaries are easier to detect. 
Overall, examining the specific effects of perturbations rather than solely focusing on spatial properties indicates that the most impactful perturbations—specifically bias, noise, and spike—tend to correlate with higher values. In the context of PET imaging, the perturbations of blur and motion exhibit negative correlations across both channels, while bias correlates negatively only for the primary channel. This suggests that generally, the lower the values of properties such as size, intensity, variability, and complexity, the more significantly the segmentation is affected by these perturbations.
The observed similarity between the spike degrees 2 and 3 across all four graphs can be attributed to the fact that, at these levels of degradation, the dice coefficient frequently approaches zero. This extreme reduction in dice value results in nearly identical delta values for these degrees, leading to a similar pattern of correlation propagation.

\newpage
\subsection{Clinical evaluation}
As shown in the pie chart, the majority of the scores are above 0.6, indicating strong performance from a metric perspective. When comparing these results with real-world gradings, we observe that most cases fall within the 3- and 4-star categories, suggesting promising clinical utility.

To further validate these findings, we examined the correlation to confirm that these scores accurately represent the underlying cases. For the original cases, the correlation was found to be in the high-moderate range, with a stronger correlation for primary tumors. This is expected, as primary tumors tend to be singular and centrally located in the images, while nodal labels can be multiple and distributed across both sides, introducing more variables that can affect the model's and physician's judgment.

Following perturbation, we observed a high correlation (~0.9) for nodal labels, which may seem significant. However, this result should be interpreted with caution, as there were several 1-star evaluations (7 out of 18 cases) where the model failed to segment any regions. These outliers are clustered closely in the graph, artificially inflating the correlation. In contrast, for primary tumors, this bias was much less pronounced, with only two instances of 1-star evaluations. The correlation for primary tumors after perturbation was similar to that of the original cases, indicating that the metrics remained consistent and relevant to real-world clinical performance, even under perturbation.

Also, something to take in account and seen with clincian, he doesn't agree everytime with the groundtruth which create bias with dice which is directly base on the ground truth. In ideal
world the ground truth and the evaluation could be performed by the same person.

These results demonstrate a higher correlation compared to the existing literature \cite{Kofler2023}. The observed correlation indicates that the Dice coefficient is relevant for real-world applications, although there is still room for improvement in its predictive accuracy.

\subsection{Limitations}
As the clinical evaluation was performed by only one observer the results could contain bias. Also with two observers, the correlation between them could be
a good aspect to compare with dice.


\newpage
\section{Conclusions}

\textit{Formulate clear conclusions which are supported by your research results.}

\endinput