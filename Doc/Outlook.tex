

\chapter{Outlook}
Integrating additional imaging modalities, such as Magnetic Resonance Imaging (MRI), into the segmentation pipeline could significantly enhance model performance, particularly in identifying necrotic tissues. The incorporation of contrast-enhanced Computed Tomography (CT) scans may further aid in distinguishing between lymph nodes and surrounding vascular structures, thereby improving the accuracy of segmentation outcomes.
\vskip1em
In light of the findings from our analysis, there is an opportunity to retrain the segmentation model by leveraging insights gained throughout this study. This may involve refining data augmentation strategies to include a broader range of perturbations, such as biases, spikes, and various forms of noise, which can enhance the model's robustness. Additionally, a comprehensive analysis of the segmentation performance with the newly introduced modalities, alongside the enhanced CT images, would provide valuable insights into how these modifications impact the segmentation accuracy and the correlations between the Dice coefficient and clinical assessments.
\vskip1em
With the segmentation model achieved, a subsequent classification model could be developed to evaluate the likelihood of extracapsular extension (ECE). This model would facilitate further analysis of its confidence levels in detecting ECE, providing clinicians with a valuable tool for improving diagnostic precision and patient management strategies.



\endinput