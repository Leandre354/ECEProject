
\chapter{Results}

\section{Robustness}
\subsection{Perturbations effects}
The following graphs illustrates the impact of perturbations on segmentation performance. 
The columns correspond to the baseline and the six different perturbations. The rows depict the delta values of 
various segmentation metrics, indicating the extent of variation in segmentation performance due to the perturbations. 
The baseline, having no delta, represents the original values for each metric. Each of the three boxplots corresponds to 
one of three levels of perturbation severity, providing a visual representation of how segmentation performance is affected 
at increasing degrees of perturbation. 
\vskip1em
The graph below represents the scenario in which perturbations are applied to nodal tumor labels in the CT modality.
\begin{figure}[ht]
    \centering
    \includegraphics[width=1\textwidth]{images/CT_Nodal.png}
    \caption{Perturbations applied to nodal tumor labels in the CT modality}
    \label{fig:three_subfigures}
\end{figure}

The graph demonstrates significant delta values in the Dice coefficient for spike, bias, and noise perturbations, 
highlighting their substantial impact on segmentation performance. Regarding the Hausdorff distance, the bias 
perturbation exhibits the most pronounced negative effect. Sensitivity, specificity, and accuracy metrics display 
behavior similar to the Dice coefficient, with considerably larger effects observed under spike, bias, and noise perturbations. 
Across the entire set of metrics, the model appears to be relatively robust to motion, blur, and ghost perturbations, 
showing minimal variation in performance.

\begin{figure}[ht]
    \centering
    \includegraphics[width=1\textwidth]{images/CT_Primary.png}
    \caption{Perturbations applied to primary tumor labels in the CT modality}
    \label{fig:three_subfigures}
\end{figure}

Here, perturbations were also applied to CT images with a focus on primary tumors. Overall, 
the behavior observed aligns with previous findings from nodal tumors, indicating that spike, noise, and bias perturbations 
are more detrimental compared to other conditions. The Hausdorff distance metric was particularly sensitive 
to bias, showing a significant impact on segmentation accuracy. Additionally, noise exhibited a less pronounced impact 
at the first degree of severity compared to higher levels for both labels, while the effects of spike and bias were pronounced 
even at this initial degree.
\begin{figure}[ht]
    \centering
    \includegraphics[width=1\textwidth]{images/PET_Nodal.png}
    \caption{Perturbations applied to primary tumor labels in the CT modality}
    \label{fig:three_subfigures}
\end{figure}
\newpage
\begin{figure}[ht]
    \centering
    \includegraphics[width=1\textwidth]{images/PET_Primary.png}
    \caption{Perturbations applied to primary tumor labels in the CT modality}
    \label{fig:three_subfigures}
\end{figure}
The two graphs show the effect of the perturbations on the PET images in both labels.

\subsection{Correlation with properties}
\begin{figure}[ht]
    \centering
    \includegraphics[width=1\textwidth]{images/CT_Nodal_Properties_1.png}
    \caption{Perturbations applied to primary tumor labels in the CT modality}
    \label{fig:three_subfigures}
\end{figure}
\begin{figure}[ht]
    \centering
    \includegraphics[width=1\textwidth]{images/CT_Nodal_Properties_2.png}
    \caption{Perturbations applied to primary tumor labels in the CT modality}
    \label{fig:three_subfigures}
\end{figure}
\begin{figure}[ht]
    \centering
    \includegraphics[width=1\textwidth]{images/CT_Nodal_Properties_3.png}
    \caption{Perturbations applied to primary tumor labels in the CT modality}
    \label{fig:three_subfigures}
\end{figure}
\begin{figure}[ht]
    \centering
    \includegraphics[width=1\textwidth]{images/CT_Nodal_Properties_4.png}
    \caption{Perturbations applied to primary tumor labels in the CT modality}
    \label{fig:three_subfigures}
\end{figure}
\begin{figure}[ht]
    \centering
    \includegraphics[width=0.6\textwidth]{images/CT_Nodal_cor.PNG}
    \caption{Perturbations applied to primary tumor labels in the CT modality}
    \label{fig:three_subfigures}
\end{figure}
\begin{figure}[ht]
    \centering
    \includegraphics[width=0.6\textwidth]{images/CT_Primary_cor.PNG}
    \caption{Perturbations applied to primary tumor labels in the CT modality}
    \label{fig:three_subfigures}
\end{figure}
\begin{figure}[ht]
    \centering
    \includegraphics[width=0.6\textwidth]{images/PET_Nodal_cor.png}
    \caption{Perturbations applied to primary tumor labels in the CT modality}
    \label{fig:three_subfigures}
\end{figure}
\begin{figure}[ht]
    \centering
    \includegraphics[width=0.6\textwidth]{images/PET_Primary_cor.png}
    \caption{Perturbations applied to primary tumor labels in the CT modality}
    \label{fig:three_subfigures}
\end{figure}
\section{Clinical evaluation}

